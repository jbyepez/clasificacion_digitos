\documentclass[11pt,twoside]{article}\usepackage[]{graphicx}\usepackage[]{color}
%% maxwidth is the original width if it is less than linewidth
%% otherwise use linewidth (to make sure the graphics do not exceed the margin)
\makeatletter
\def\maxwidth{ %
  \ifdim\Gin@nat@width>\linewidth
    \linewidth
  \else
    \Gin@nat@width
  \fi
}
\makeatother

\definecolor{fgcolor}{rgb}{0.345, 0.345, 0.345}
\newcommand{\hlnum}[1]{\textcolor[rgb]{0.686,0.059,0.569}{#1}}%
\newcommand{\hlstr}[1]{\textcolor[rgb]{0.192,0.494,0.8}{#1}}%
\newcommand{\hlcom}[1]{\textcolor[rgb]{0.678,0.584,0.686}{\textit{#1}}}%
\newcommand{\hlopt}[1]{\textcolor[rgb]{0,0,0}{#1}}%
\newcommand{\hlstd}[1]{\textcolor[rgb]{0.345,0.345,0.345}{#1}}%
\newcommand{\hlkwa}[1]{\textcolor[rgb]{0.161,0.373,0.58}{\textbf{#1}}}%
\newcommand{\hlkwb}[1]{\textcolor[rgb]{0.69,0.353,0.396}{#1}}%
\newcommand{\hlkwc}[1]{\textcolor[rgb]{0.333,0.667,0.333}{#1}}%
\newcommand{\hlkwd}[1]{\textcolor[rgb]{0.737,0.353,0.396}{\textbf{#1}}}%
\let\hlipl\hlkwb

\usepackage{framed}
\makeatletter
\newenvironment{kframe}{%
 \def\at@end@of@kframe{}%
 \ifinner\ifhmode%
  \def\at@end@of@kframe{\end{minipage}}%
  \begin{minipage}{\columnwidth}%
 \fi\fi%
 \def\FrameCommand##1{\hskip\@totalleftmargin \hskip-\fboxsep
 \colorbox{shadecolor}{##1}\hskip-\fboxsep
     % There is no \\@totalrightmargin, so:
     \hskip-\linewidth \hskip-\@totalleftmargin \hskip\columnwidth}%
 \MakeFramed {\advance\hsize-\width
   \@totalleftmargin\z@ \linewidth\hsize
   \@setminipage}}%
 {\par\unskip\endMakeFramed%
 \at@end@of@kframe}
\makeatother

\definecolor{shadecolor}{rgb}{.97, .97, .97}
\definecolor{messagecolor}{rgb}{0, 0, 0}
\definecolor{warningcolor}{rgb}{1, 0, 1}
\definecolor{errorcolor}{rgb}{1, 0, 0}
\newenvironment{knitrout}{}{} % an empty environment to be redefined in TeX

\usepackage{alltt}
\usepackage[spanish, es-tabla]{babel}
\usepackage[utf8]{inputenc}
\IfFileExists{upquote.sty}{\usepackage{upquote}}{}
\begin{document}
\SweaveOpts{concordance=TRUE}


\begin{abstract}
\noindent
En este trabajo se aborada la construccion de modelos de clasificación multiclase usando el conjutno de datos MNIST  que es una gran base de datos de dígitos escritos a mano.
las metodologias que se abordaran para contruir los modelos son las siguientes
1. Regresión logística multinomial\\
2. Árboles de clasificación\\
3. Bosques aleatorios\\
4. Máquinas de soporte vectorial\\
5. Redes neuronales\\
En este desarrollo se comprara el rendimiento de los modelos antes mensionados para verdificar cual es la mejor de estas metodologias con este conjutno de datos particular.  
\end{abstract}


\section{Definiciones}

MINIST\\
La base de datos MNIST (base de datos modificada del Instituto Nacional de Estándares y Tecnología ) es una gran base de datos de dígitos escritos a mano que se usa comúnmente para capacitar a varios sistemas de procesamiento de imágenes.La base de datos también se usa ampliamente para capacitación y pruebas en el campo del aprendizaje automático.

Modelo Estadístico\\
El modelado estadístico es una forma simplificada, matemáticamente formalizada, de aproximarse a la realidad, la que genera los datos) y, opcionalmente, hacer predicciones a partir de dicha aproximación.

\section{Metodología}
\subsection{Análisis descriptivo}
\begin{knitrout}
\definecolor{shadecolor}{rgb}{0.969, 0.969, 0.969}\color{fgcolor}\begin{kframe}
\begin{alltt}
\hlstd{trainingData} \hlkwb{<-} \hlkwd{read.csv}\hlstd{(}\hlstr{"../data/mnist_train.csv"}\hlstd{,}\hlkwc{header} \hlstd{=} \hlnum{FALSE}\hlstd{)}
\hlkwd{table}\hlstd{(trainingData}\hlopt{$}\hlstd{V1)}
\end{alltt}
\begin{verbatim}
## 
##    0    1    2    3    4    5    6    7    8    9 
## 5923 6742 5958 6131 5842 5421 5918 6265 5851 5949
\end{verbatim}
\end{kframe}
\end{knitrout}



\subsection{Modelamiento estadístico}

\subsection{Diagnóstico del Modelo}
\section{Conclusiones}
\section{Referencias}
\begin{description}
\end{description}

\end{document}
